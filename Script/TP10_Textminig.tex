% Options for packages loaded elsewhere
\PassOptionsToPackage{unicode}{hyperref}
\PassOptionsToPackage{hyphens}{url}
%
\documentclass[
]{article}
\usepackage{amsmath,amssymb}
\usepackage{iftex}
\ifPDFTeX
  \usepackage[T1]{fontenc}
  \usepackage[utf8]{inputenc}
  \usepackage{textcomp} % provide euro and other symbols
\else % if luatex or xetex
  \usepackage{unicode-math} % this also loads fontspec
  \defaultfontfeatures{Scale=MatchLowercase}
  \defaultfontfeatures[\rmfamily]{Ligatures=TeX,Scale=1}
\fi
\usepackage{lmodern}
\ifPDFTeX\else
  % xetex/luatex font selection
\fi
% Use upquote if available, for straight quotes in verbatim environments
\IfFileExists{upquote.sty}{\usepackage{upquote}}{}
\IfFileExists{microtype.sty}{% use microtype if available
  \usepackage[]{microtype}
  \UseMicrotypeSet[protrusion]{basicmath} % disable protrusion for tt fonts
}{}
\makeatletter
\@ifundefined{KOMAClassName}{% if non-KOMA class
  \IfFileExists{parskip.sty}{%
    \usepackage{parskip}
  }{% else
    \setlength{\parindent}{0pt}
    \setlength{\parskip}{6pt plus 2pt minus 1pt}}
}{% if KOMA class
  \KOMAoptions{parskip=half}}
\makeatother
\usepackage{xcolor}
\usepackage[margin=1in]{geometry}
\usepackage{graphicx}
\makeatletter
\def\maxwidth{\ifdim\Gin@nat@width>\linewidth\linewidth\else\Gin@nat@width\fi}
\def\maxheight{\ifdim\Gin@nat@height>\textheight\textheight\else\Gin@nat@height\fi}
\makeatother
% Scale images if necessary, so that they will not overflow the page
% margins by default, and it is still possible to overwrite the defaults
% using explicit options in \includegraphics[width, height, ...]{}
\setkeys{Gin}{width=\maxwidth,height=\maxheight,keepaspectratio}
% Set default figure placement to htbp
\makeatletter
\def\fps@figure{htbp}
\makeatother
\setlength{\emergencystretch}{3em} % prevent overfull lines
\providecommand{\tightlist}{%
  \setlength{\itemsep}{0pt}\setlength{\parskip}{0pt}}
\setcounter{secnumdepth}{-\maxdimen} % remove section numbering
\usepackage{hyperref}
\usepackage{amsmath}
\usepackage{amssymb}
\usepackage{graphicx}
\usepackage{fontspec}
\usepackage{xcolor}
\usepackage{tikz}
\definecolor{green}{RGB}{0, 100, 0}
\setmainfont{Times New Roman}
\setsansfont{Times New Roman}
\setmonofont{Courier New}
\usepackage[margin=1in]{geometry}
\usepackage{titlesec}
\titleformat{\section}{\Huge\bfseries\color{green}}{\thesection}{1em}{}
\titleformat{\subsection}{\huge\bfseries\color{green}}{\thesubsection}{1em}{}
\titleformat{\subsubsection}{\LARGE\bfseries\color{green}}{\thesubsubsection}{1em}{}
\usepackage{tocloft}
\renewcommand{\cftsecfont}{\small}
\renewcommand{\cftsubsecfont}{\footnotesize}
\renewcommand{\cftsecpagefont}{\small}
\renewcommand{\cftsubsecpagefont}{\footnotesize}
\renewcommand{\cftsecleader}{\cftdotfill{\cftdotsep}}
\ifLuaTeX
  \usepackage{selnolig}  % disable illegal ligatures
\fi
\usepackage{bookmark}
\IfFileExists{xurl.sty}{\usepackage{xurl}}{} % add URL line breaks if available
\urlstyle{same}
\hypersetup{
  hidelinks,
  pdfcreator={LaTeX via pandoc}}

\author{}
\date{\vspace{-2.5em}}

\begin{document}

\begin{titlepage}
    \begin{center}
    % Début de la bordure avec TikZ
    \begin{tikzpicture}[remember picture, overlay] % overlay = force TikZ à dessiner par-dessus le contenu existant de la page, plutôt que de réserver un espace supplémentaire pour le dessin, remember = lorsque vous voulez dessiner par rapport à la position exacte de la page
    % Définir une couleur élégante
    \definecolor{green}{RGB}{0, 100, 0}
    
    % Dessiner une bordure élégante avec coins arrondis
    \draw[
        line width=5pt, % Épaisseur du trait
        green, % Couleur de la bordure
        rounded corners=15pt, % Coins arrondis
        double, % Bordure double
        double distance=2pt % Espacement entre les deux lignes
        ]
        
        ([xshift=10pt, yshift=-10pt]current page.north west) rectangle
        ([xshift=-10pt, yshift=10pt]current page.south east);
    \end{tikzpicture}
        \includegraphics[width=7cm]{images/LOGO1.png} \\[0.1cm]  
        
        \includegraphics[width=6cm]{images/LOGO2.png} \\[0.1cm] 
        
        \textbf{\large Agence nationale de la Statistique et de la Démographie (ANSD)}\\[0.2cm]
        
        \includegraphics[width=4cm]{images/LOGO3.png} \\[0.1cm]
        
        \textbf{\large Ecole nationale de la Statistique et de l'Analyse économique Pierre Ndiaye (ENSAE)}\\[0.4cm]
        
        \textit{\LARGE Semestre 2 :PROJET STATISTIQUES AVEC R }\\[0.3cm]
        \textbf{\Huge \color{green} \textsf{TP 10 : Traitement des questions ouvertes avec R : Texte mining}}\\[0.2cm]
        
        \begin{minipage}{0.5\textwidth}
    \begin{flushleft} \large
        \emph{\textsf{Rédigé par :}}\\
        \textbf{Paul BALAFAI}\\
        \textbf{Mame Balla BOUSSO}\\
        \textit{Elèves ingénieurs statisticiens économistes}
    \end{flushleft}
\end{minipage}
        \hfill
        \begin{minipage}{0.4\textwidth}
            \begin{flushright} \large
                \emph{\textsf{Sous la supervision de :}} \\
                \textbf{M. Aboubacre HEMA}\\
                \textit{Research Analyst }
            \end{flushright}
        \end{minipage}

        \vfill 

        {\large \textsf{Année scolaire : 2024/2025}}\\[0.5cm]
        
    \end{center}
\end{titlepage}

\subsection{Sommaire}\label{sommaire}

\begin{itemize}
\tightlist
\item
  \hyperref[introduction-uxe0-r-markdown]{Introduction à R Markdown}
\item
  \hyperref[1-installation-et-configuration-de-rmarkdown]{Installation
  et configuration}
\item
  \hyperref[2-uxe9crire-un-document-r-markdown]{Écrire un document R
  Markdown}
\item
  \hyperref[3-Bonnes-pratiques-dans-R-Markdown]{Bonnes pratiques dans R
  Markdown}
\item
  \hyperref[ruxe9fuxe9rences-bibliographiques]{Références
  bibliographiques}
\end{itemize}

\newpage

\section{Introduction à R Markdown}\label{introduction-uxe0-r-markdown}

\end{document}
